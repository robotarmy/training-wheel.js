\documentclass[14pt]{book}
\usepackage{fontspec} %xelatex or lualatex
\usepackage{graphicx}
\usepackage{float}

\usepackage{color}
\usepackage{xcolor}
\usepackage{calc}
\newlength\captionLeftDimension
\newlength\captionRightDimension
\setlength\captionLeftDimension{ \fboxsep+(\fboxrule*3)}
\setlength\captionRightDimension{-\fboxsep+(\fboxrule*3)}
\usepackage{listings}

\definecolor{lightgray}{rgb}{.929,.929,.929}
\definecolor{darkgray}{rgb}{.4,.4,.4}
\definecolor{purple}{rgb}{0.65, 0.12, 0.82}

%http://lenaherrmann.net/2010/05/20/javascript-syntax-highlighting-in-the-latex-listings-package
\lstdefinelanguage{JavaScript}{
  keywords={typeof, new, true, false, catch, function, return, null, catch, switch, var, if, in, while, do, else, case, break},
  keywordstyle=\color{blue}\bfseries,
  ndkeywords={class, export, boolean, throw, implements, import, this},
  ndkeywordstyle=\color{darkgray}\bfseries,
  identifierstyle=\color{black},
  sensitive=false,
  comment=[l]{//},
  morecomment=[s]{/*}{*/},
  commentstyle=\color{purple}\ttfamily,
  stringstyle=\color{red}\ttfamily,
  morestring=[b]',
  morestring=[b]"
}

\lstset{
   language=JavaScript,
   backgroundcolor=\color{lightgray},
   extendedchars=true,
   basicstyle=\footnotesize\ttfamily,
   showstringspaces=false,
   showspaces=false,
   numbers=left,
   numberstyle=\footnotesize,
   numbersep=9pt,
   tabsize=2,
   breaklines=true,
   showtabs=false,
   captionpos=b
}

\lstset{frame=single,
    frameround=fttf,
    xleftmargin=\captionLeftDimension,
    rulecolor=\color{gray},
    xrightmargin=\captionRightDimension}
\usepackage{caption}
\DeclareCaptionFont{white}{\color{white}}
\DeclareCaptionFormat{listing}{\parbox{\textwidth}{\colorbox{gray}{\parbox{\textwidth}{#1#2#3}}\vskip+1pt}}
\captionsetup[lstinputlisting]{format=listing,labelfont=white,textfont=white}



%http://www.maths.manchester.ac.uk/~kd/latextut/pdfbyex.htm
%http://www.maths.manchester.ac.uk/~kd/latextut/bookex.pdf
\parindent 1cm
\parskip 0.2cm
\topmargin 0.2cm
\oddsidemargin 1cm
\evensidemargin 0.5cm
\textwidth 15cm
\textheight 21cm

% rewrite section into newpage and section
%\let\stdsection\section
%\renewcommand\section{\newpage\stdsection}

%\makeindex -- how to use this?
% Note the tag used to make an index entry. You may need to consult Lamport's
% book~\cite{lamport} for details of the procedure to make the index input
% file; \LaTeX \ will create a pre-index by listing all the tagged
% items in the file {\tt bookex.idx} then you edit this into
% a {\tt theindex} environment, as {\tt index.tex}.


\title{Wheel.js By Example  }

\author{Curtis J. Schofield \\
{\small\em \copyright \  Draft date \today }}

\begin{document}
% \maketitle -- how to use this?
 \addcontentsline{toc}{chapter}{Contents}
\pagenumbering{arabic}
\tableofcontents
%\listoffigures
%\listoftables
\chapter*{Preface}\normalsize
  \addcontentsline{toc}{chapter}{Preface}
\pagestyle{plain}
This book provides a way to learn {\tt Wheel.js} by exploring it's
contepts and providing examples of common {\tt Wheel.js } patterns
and implementation of common interface interaction patters with {\tt
Wheel.js }.




\pagestyle{headings}
\pagenumbering{arabic}

\chapter{Step by Step Concepts of Wheel.js}
\begin{center}
{\small\em To illuminate the concepts of Wheel.js}
\end{center}


\section{What is Wheel.js?} % (fold)
\label{sec:What is Wheel.js?}

% section What is Wheel.js? (end)

\section{What is an App?} % (fold)
\label{sec:What is an App?}

% section What is an App? (end)

\section{Creating a View} % (fold)
\label{sec:Creating a View}

% section Creating a View (end)

\section{What is dollar?} % (fold)
\label{sec:What is dollar?}

% section What is dollar? (end)

\section{Simple Event Handling} % (fold)
\label{sec:Simple Event Handling}

% section Simple Event Handling (end)

\section{Supported Events} % (fold)
\label{sec:Supported Events}



% section Supported Events (end)
\chapter{Exapmles of Wheel.js}
\begin{center}
{\small\em To provide examples of Wheel.js}
\end{center}

\section{Smooth Drag Event Example} % (fold)
\label{sec:Smooth Drag Event Example}

 Implementing Drag Events in Wheel involves 2 stages.

 The first stage is to know what kind of event you would like to initiate the drag.
 In my exapmle I am using 'mousedown' and 'touchstart' - this gives the effect of 'click-and-hold'
 for drag on desktop and the effect of touch-and-hold on touch devices.

 The second stage is to know that wheel understands 3 events related to drag.
 \begin{description}
   \item[draginit] It is recommended to trigger this event in the initating 
                   event of the 1st stage.
   \item[dragmove] Once draginit has been triggered - dragmove events 
                   will be generated on movement.
   \item[dragend]  This is automatically called when Wheel detects the 
                   opposite of your initate drag event - in the case of our example
                   it will be 'mouseup' and 'touchend'.
 \end{description}

\definecolor{javared}{rgb}{0.6,0,0} % for strings
\definecolor{javagreen}{rgb}{0.25,0.5,0.35} % comments
\definecolor{javapurple}{rgb}{0.5,0,0.35} % keywords
\definecolor{javadocblue}{rgb}{0.25,0.35,0.75} % javadoc

\lstset{language=Java,
basicstyle=\ttfamily,
keywordstyle=\color{javapurple}\bfseries,
stringstyle=\color{javared},
commentstyle=\color{javagreen},
morecomment=[s][\color{javadocblue}]{/**}{*/},
numbers=left,
numberstyle=\tiny\color{black},
stepnumber=2,
numbersep=10pt,
tabsize=4,
showspaces=false,
showstringspaces=false}
 
\definecolor{javared}{rgb}{0.6,0,0} % for strings
\definecolor{javagreen}{rgb}{0.25,0.5,0.35} % comments
\definecolor{javapurple}{rgb}{0.5,0,0.35} % keywords
\definecolor{javadocblue}{rgb}{0.25,0.35,0.75} % javadoc

\lstset{
basicstyle=\ttfamily,
keywordstyle=\color{javapurple}\bfseries,
stringstyle=\color{javared},
commentstyle=\color{javagreen},
morecomment=[s][\color{javadocblue}]{/**}{*/},
numbers=left,
numberstyle=\tiny\color{black},
stepnumber=2,
numbersep=10pt,
tabsize=4,
showspaces=false,
breaklines=true,
showstringspaces=false}

 \lstinputlisting{../examples/7-event-area-app.html}
 
 
\begin{figure}[H] % the H is what makes it do a strong 'here' - float is is used for 'H'
\begin{center}
\includegraphics[height=1\textwidth]{figures/drag-event-example-1.mps}
\end{center}
  \caption{Implementing Smooth Drag in Wheel.js}
\end{figure}
% section Smooth Drag Event Example (end)





\begin{thebibliography}{99}
  \addcontentsline{toc}{chapter}{Bibliography}
\bibitem{lamport} L. Lamport. {\bf \LaTeX \ A Document Preparation System}
Addison-Wesley, California 1986.
\bibitem{lamport} CTJ. Dodson. {\bf \LaTeX \ PDF \LaTeX \ by Example}
http://www.maths.manchester.ac.uk/~kd/latextut/pdfbyex.htm, 2001.
\end{thebibliography}


\end{document}


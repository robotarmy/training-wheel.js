\documentclass[16pt]{book}
%\usepackage{fontspec} %xelatex or lualatex
\usepackage{graphicx}
\usepackage{float}
%http://www.maths.manchester.ac.uk/~kd/latextut/pdfbyex.htm
%http://www.maths.manchester.ac.uk/~kd/latextut/bookex.pdf
\parindent 1cm
\parskip 0.2cm
\topmargin 0.2cm
\oddsidemargin 1cm
\evensidemargin 0.5cm
\textwidth 15cm
\textheight 21cm

% rewrite section into newpage and section
%\let\stdsection\section
%\renewcommand\section{\newpage\stdsection}

%\makeindex -- how to use this?


\title{Wheel.js By Example  }

\author{Kane _ _ \& Curtis J. Schofield \\
{\small\em \copyright \  Draft date \today }}

\begin{document}
% \maketitle -- how to use this?
 \addcontentsline{toc}{chapter}{Contents}
\pagenumbering{arabic}
\tableofcontents
%\listoffigures
%\listoftables
\chapter*{Preface}\normalsize
  \addcontentsline{toc}{chapter}{Preface}
\pagestyle{plain}
The book root file {\tt bookex.tex} gives a basic example of how to
use \LaTeX \ for preparation of a book. Note that all
\LaTeX \ commands begin with a
backslash.

Each
Chapter, Appendix and the Index is made as a {\tt *.tex} file and is
called in by the {\tt include} command---thus {\tt ch1.tex} is
the name here of the file containing Chapter~1. The inclusion of any
particular file can be suppressed by prefixing the line by a
percent sign.


 Do not put an {\tt end{document}} command at the end of chapter files;
just one such command is needed at the end of the book.

Note the tag used to make an index entry. You may need to consult Lamport's
book~\cite{lamport} for details of the procedure to make the index input
file; \LaTeX \ will create a pre-index by listing all the tagged
items in the file {\tt bookex.idx} then you edit this into
a {\tt theindex} environment, as {\tt index.tex}.





\pagestyle{headings}
\pagenumbering{arabic}

%\include{ch1}
%\include{ch2}





\chapter{Step by Step Basics of Wheel.js}
\begin{center}
{\small\em To illuminate the concepts of Wheel.js}
\end{center}


\section{What is Wheel.js?} % (fold)
\label{sec:What is Wheel.js?}

% section What is Wheel.js? (end)

\section{What is an App?} % (fold)
\label{sec:What is an App?}

% section What is an App? (end)

\section{Creating a View} % (fold)
\label{sec:Creating a View}

% section Creating a View (end)

\section{What is dollar?} % (fold)
\label{sec:What is dollar?}

% section What is dollar? (end)

\section{Simple Event Handling} % (fold)
\label{sec:Simple Event Handling}

% section Simple Event Handling (end)

\section{Supported Events} % (fold)
\label{sec:Supported Events}

\newpage
% section Supported Events (end)
\chapter{Exapmles of Wheel.js}
\begin{center}
{\small\em To provide examples of Wheel.js}
\end{center}

\section{Smooth Drag Event Example} % (fold)
\label{sec:Smooth Drag Event Example}

\begin{figure}[H] % the H is what makes it do a strong 'here' - float is is used for 'H'
\begin{center}
\includegraphics[height=1\textwidth]{figures/drag-event-example-1.mps}
\end{center}
  \caption{Implementing Smooth Drag in Wheel.js}
\end{figure}
% section Smooth Drag Event Example (end)
\newpage

\begin{thebibliography}{99}
  \addcontentsline{toc}{chapter}{Bibliography}
\bibitem{lamport} L. Lamport. {\bf \LaTeX \ A Document Preparation System}
Addison-Wesley, California 1986.
\bibitem{lamport} CTJ. Dodson. {\bf \LaTeX \ PDF \LaTeX \ by Example}
http://www.maths.manchester.ac.uk/~kd/latextut/pdfbyex.htm, 2001.
\end{thebibliography}


\end{document}


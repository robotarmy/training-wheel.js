
% section Supported Events (end)
\chapter{Exapmles of Wheel.js}
\begin{center}
{\small\em To provide examples of Wheel.js}
\end{center}

\section{Smooth Drag Event Example} % (fold)
\label{sec:Smooth Drag Event Example}

 Implementing Drag Events in Wheel involves 2 stages.

 The first stage is to know what kind of event you would like to initiate the drag.
 In my exapmle I am using 'mousedown' and 'touchstart' - this gives the effect of 'click-and-hold'
 for drag on desktop and the effect of touch-and-hold on touch devices.

 The second stage is to know that wheel understands 3 events related to drag.
 \begin{description}
   \item[draginit] It is recommended to trigger this event in the initating 
                   event of the 1st stage.
   \item[dragmove] Once draginit has been triggered - dragmove events 
                   will be generated on movement.
   \item[dragend]  This is automatically called when Wheel detects the 
                   opposite of your initate drag event - in the case of our example
                   it will be 'mouseup' and 'touchend'.
 \end{description}

\definecolor{javared}{rgb}{0.6,0,0} % for strings
\definecolor{javagreen}{rgb}{0.25,0.5,0.35} % comments
\definecolor{javapurple}{rgb}{0.5,0,0.35} % keywords
\definecolor{javadocblue}{rgb}{0.25,0.35,0.75} % javadoc

\lstset{language=Java,
basicstyle=\ttfamily,
keywordstyle=\color{javapurple}\bfseries,
stringstyle=\color{javared},
commentstyle=\color{javagreen},
morecomment=[s][\color{javadocblue}]{/**}{*/},
numbers=left,
numberstyle=\tiny\color{black},
stepnumber=2,
numbersep=10pt,
tabsize=4,
showspaces=false,
showstringspaces=false}
 
\definecolor{javared}{rgb}{0.6,0,0} % for strings
\definecolor{javagreen}{rgb}{0.25,0.5,0.35} % comments
\definecolor{javapurple}{rgb}{0.5,0,0.35} % keywords
\definecolor{javadocblue}{rgb}{0.25,0.35,0.75} % javadoc

\lstset{
basicstyle=\ttfamily,
keywordstyle=\color{javapurple}\bfseries,
stringstyle=\color{javared},
commentstyle=\color{javagreen},
morecomment=[s][\color{javadocblue}]{/**}{*/},
numbers=left,
numberstyle=\tiny\color{black},
stepnumber=2,
numbersep=10pt,
tabsize=4,
showspaces=false,
breaklines=true,
showstringspaces=false}

 \lstinputlisting{../examples/7-event-area-app.html}
 
 
\begin{figure}[H] % the H is what makes it do a strong 'here' - float is is used for 'H'
\begin{center}
\includegraphics[height=1\textwidth]{figures/drag-event-example-1.mps}
\end{center}
  \caption{Implementing Smooth Drag in Wheel.js}
\end{figure}
% section Smooth Drag Event Example (end)


% section Supported Events (end)
\chapter{Examples of Wheel.js}
\begin{center}
{\small\em To provide examples of Wheel.js}
\end{center}

\section{Smooth Drag Event Example} % (fold)
\label{sec:Smooth Drag Event Example}

 Implementing Drag Events in Wheel involves setting up listeners which will
 fire in 2 stages.

 \lstinputlisting[firstnumber=60,firstline=60,lastline=73]{../examples/6-patch-drag-app.html}

 The first stage is to know what kind of event you would like to initiate the drag.

 In my example I have chosen \emph{tapstart} (which is both 'mousedown' and 'touchstart') 
 this gives the effect of 'click-and-hold' for drag on desktop and
 the effect of 'touch-and-hold' on touch devices.

 The second stage is to know that wheel understands 4 events related to drag.

 \begin{itemize}
   \item[draginit]
     \begin{enumerate}
       \item This is an event that is triggered in the callback event of
                   and initiating event. The callback snippet below demonstrates 
                   using the \emph{\$} dom-property to invoke the \emph{draginit} event.
                  \lstinputlisting[firstnumber=75,firstline=75,lastline=78]{../examples/6-patch-drag-app.html}
     \end{enumerate}
   \item[dragstart] 
 \begin{enumerate}
   \item Once \emph{draginit} has been triggered - 
                    \emph{dragstart} will be called by the Wheel framework.
   \lstinputlisting[firstnumber=80,firstline=80,lastline=83]{../examples/6-patch-drag-app.html}
 \end{enumerate}
   \item[dragmove] 
     \begin{enumerate}
       \item The Wheel framework will generate \emph{dragmove} events on movement of the still-touching
                   finger - or the still held down mouse. The position of the finger
                   or mouse respectively is encoded in 'pageX' and 'pageY'.
   \lstinputlisting[firstnumber=85,firstline=85,lastline=87]{../examples/6-patch-drag-app.html}
     \end{enumerate}
   \item[dragend]
    \begin{enumerate}
      \item This is triggered when Wheel detects the 
                   opposite of your initiate drag event. In this case it will be 
                   \emph{touchstop}
    \lstinputlisting[firstnumber=89,firstline=89,lastline=92]{../examples/6-patch-drag-app.html}

    \end{enumerate}
\end{itemize}


 I am going to explain \emph{dragstart}, \emph{dragmove} and \emph{dragend} in terms of STARTING, DYNAMIC, and FINAL states
 and how they can be used to implement a smooth drag algorithm.
 To ease explaining
 I will refer the the mobile interaction with the touching finger - on a mobile device the finger shares most
 properties of a desktop mouse being moved with a left-click held. 

 In the below diagram I have detailed these 3 'states' - STARTING, DYNAMIC, and FINAL -- I will explain
 them following the diagram.

\begin{figure}[H] % the H is what makes it do a strong 'here' - float is is used for 'H'
\begin{center}
\includegraphics[height=1\textwidth]{figures/drag-event-example-1.mps}
\end{center}
  \caption{Implementing Smooth Drag in Wheel.js}
\end{figure}

 \begin{description}
   \item[STARTING] state is used to record the initial touch origin of the finger press. These inital values
     are used as the basis for calculating 'horizontal travel' and 'vertical travel'. In wheel this state is
     realized by using \emph{dragstart}
     
   \lstinputlisting[firstnumber=80,firstline=80,lastline=83]{../examples/6-patch-drag-app.html}
   \item[DYNAMIC] state is when the finger is held on the screen and moved around. 
     In wheel this state is realized by using \emph{dragmove}

     Wheel normalizes what would generally be clientX/clientY or pageX/pageY in various browsers into pageX/pageY. 
   \lstinputlisting[firstnumber=85,firstline=85,lastline=87]{../examples/6-patch-drag-app.html}
   \textbf{Wheel also has the faculty to provide deltaX and deltaY properties as part of an event}.
      Converting this
     code to use this simplifying faculty of Wheel is left as an excersize to the reader.
     
     For the purposes of this example we are calculating deltaX and deltaY manually to expose
     the relationships that can be utilized by harnessing \emph{dragstart}, \emph{dragmove} and \emph{dragend}
     together. I refer to deltaX as horizontalTravel and deltaY as verticalTravel in the example.
     
     This function details the process of calculating horizontalTravel and verticalTravel from the value of pageX
     and pageY. The horizontal travel and the vertical travel are relative to the touch position recorded in \emph{dragstart}.
   \lstinputlisting[firstnumber=94,firstline=94,lastline=97]{../examples/6-patch-drag-app.html}
     The current postion of the box that the finger is moving is based on the value that is stored in the View object.
     The initial values are 0 and 0 for x and y respectively. This current position becomes the next position by including
     the values of horizontal travel and vertical travel to x and y respectively.
   \lstinputlisting[firstnumber=98,firstline=98,lastline=100]{../examples/6-patch-drag-app.html}
     The \$ operator is used to alter the css style attribute with this next value for top and left. This creates the
     illusion that the box is moving with the finger.
   \lstinputlisting[firstnumber=101,firstline=101,lastline=103]{../examples/6-patch-drag-app.html}
   \item[FINAL] state is when the finger is released. In wheel this state is realized by using \emph{dragend}.
     The values from the previous state dynamic will be commited to the object as the current position of the box that
     the view represents. Because the dom has already been updated in the DYNAMIC state - what you see is what the
     object knows as-well.
    \lstinputlisting[firstnumber=89,firstline=89,lastline=92]{../examples/6-patch-drag-app.html}

 \end{description}

 \newpage
 The complete code listings for this example is as follows :
 \lstinputlisting{../examples/6-patch-drag-app.html}
% section Smooth Drag Event Example (end)

